\secnumbersection{CONCLUSIONES}

Las Conclusiones son, según algunos especialistas, el aspecto principal de una memoria, ya que reflejan el aprendizaje final del autor del documento. En ellas se tiende a considerar los alcances y limitaciones de la propuesta de solución, establecer de forma simple y directa los resultados, discutir respecto a la validez de los objetivos formulados, identificar las principales contribuciones y aplicaciones del trabajo realizado, así como su impacto o aporte a la organización o a los actores involucrados. Otro aspecto que tiende a incluirse son recomendaciones para quienes se sientan motivados por el tema y deseen profundizarlo, o lineamientos de una futura ampliación del trabajo.

\underline{Todo esto debe sintetizarse en al menos 5 páginas.}

FIXME: CONCLUSIONES P1

\newpage
FIXME: CONCLUSIONES P2

\newpage
FIXME: CONCLUSIONES P3

\newpage
FIXME: CONCLUSIONES P4

\newpage
FIXME: CONCLUSIONES P5
