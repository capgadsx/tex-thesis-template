\documentclass[conference,compsoc]{IEEEtran}
\usepackage{graphicx}
\graphicspath{ {figures/} }
\usepackage[spanish,es-tabla]{babel}
\usepackage[T1]{fontenc}
\usepackage{fontspec}
\usepackage{anyfontsize}
% \usepackage[letterpaper,top=2.5cm,bottom=3cm,left=3cm,right=3cm,marginparwidth=1.75cm]{geometry}
% \renewcommand{\baselinestretch}{1.0}
% \setlength{\parindent}{0cm}
% \setlength{\parskip}{0.4cm}
\usepackage{hyperref}
\hypersetup{colorlinks = false}

\begin{document}

\title{AUTOCOMPLETADO DE CONSULTAS SPARQL,\\UNA ALTERNATIVA OPEN SOURCE\\ \ \\PROPUESTA}

\author{
    \IEEEauthorblockN{Carlos Andrés Ponce Godoy}
    \IEEEauthorblockA{\textit{Departamento de Informática} \\
    \textit{Universidad Técnica Federico Santa María}\\
    Valparaíso, Chile\\
    carlos.ponce@sansano.usm.cl}
}

\maketitle

\IEEEpubidadjcol

\begin{abstract}
    Here goes your abstract
\end{abstract}

\begin{IEEEkeywords}
    SPARQL, Web Semántica, RDF
\end{IEEEkeywords}

\IEEEpeerreviewmaketitle

\section{Definición del problema}

Actualmente, si una persona quisiera realizar una consulta a un endpoint SPARQL,
necesita conocer dos cosas: primero, la sintaxis para escribir una consulta
SPARQL funcional y segundo, la estructura de la información almacenada en la base
de datos del endpoint, la cual se encuentra descrita a través de grafos.

Para usuarios que nunca han utilizado en lenguaje de consultas SPARQL, obtener
información desde una base de datos RDF como Wikidata, es una tarea que implica
una cantidad no menor de esfuerzo. Es por esto, que se han desarrollado herramientas
que facilitan el aprendizaje del lenguaje de consultas, como por ejemplo
RDFExplorer, [...] (investigar sobre más).

Estas herramientas nos permiten liberarnos de una de nuestras necesidades o incluso
ambas para construir nuestra consulta SPARQL. En la siguiente tabla comparativa,
podemos revisar las características y funcionalidades de las herramientas mencionadas.

[INSERTAR TABLA AQUI]

En base a esta comparación, se ha decidido trabajar con RDFExplorer, debido a su
facilidad de usó, la capacidad de construir consultas de forma interactiva, 
la posibilidad de navegar las entidades obtenidas y el poder obtener resultados
mientras se construye la consulta.

[IMAGEN DE RDF EXPLORER]

Sin embargo, aun cuando utilizamos herramientas como RDFExplorer, debemos tener
algún grado de conocimiento sobre las propiedades de las entidades disponibles
en la base de datos, conocimiento que no tienen los usuarios que utilizan por primera
vez este tipo de consultas.

Una forma de apoyar a los usuarios en la tarea de descubrir las propiedades disponibles
en la base de datos que están utilizando es generando recomendaciones o sugerencias
sobre cual propiedad se puede obtener para una determinada entidad, similar a lo
realizado por herramientas de autocompletado al escribir documentos o lo que es más
cercano al campo de la programación y la informática, el autocompletado de código en
entornos de desarrollo integrados.

Actualmente existen soluciones que generan recomendaciones en base a endpoints SPARQL
y que incluso pueden integrarse a RDFExplorer.

[OTRA TABLA COMPARATIVA]

De estas alternativas, la más prometedora para los usuarios es SPARQLForHumans, puesto
que es posible integrarlo con RDFExplorer. Sin embargo, esta solución tiene un
problema fundamental y es que la implementación se ha realizado utilizando el
framework .NET para el lenguaje de programación $C\#$, el cual, si bien actualmente los
componentes del framework se encuentran bajo las licencias MIT o Apache 2, la propiedad
sigue perteneciendo a Microsoft, lo cual no logra convencer por completo a la comunidad
open source para utilizar esta herramienta de desarrollo (generar más argumentos para esto).

En base a esto, lo que buscamos realizar con este trabajo es lograr la funcionalidad obtenida por
el proyecto SPARQLForHumans pero con una restricción importante, nuestro stack de 
tecnologías solo puede estar compuesto por software gratuito y de código abierto.
Además, buscamos realizar las siguientes mejoras al rendimiento del sistema:

\begin{itemize}
    \item Reducir los tiempos de procesamiento para los dumps de wikidata.
    \item Reducir el tiempo necesario para generar una recomendación.
\end{itemize}

\section{Objetivos}

    \subsection{Objetivo general}

Diseñar e implementar un sistema que permita obtener sugerencias para las posibles
entidades y propiedades de una consulta SPARQL incompleta utilizando exclusivamente
componentes de código abierto.

    \subsection{Objetivos especificos}

\begin{itemize}
    \item Diseñar el mecanismo de recomendación para las entidades de una consulta.
    \item Diseñar la arquitectura del sistema utilizando solamente componentes de código abierto.
    \item Implementar la arquitectura propuesta.
    \item Evaluar el rendimiento de la solución implementada.
\end{itemize}

\section{Discusión bibliográfica preliminar}

Para abordar nuestro problema primero debemos aclarar algunos conceptos que han sido
mencionados en las secciones anteriores, tales como la web semántica, SPARQL, RDF y el
contexto en el que estos se utilizan. Para esto, nos apoyaremos en las siguientes definiciones.

    \subsection{La web semántica}

La red informática mundial, \textit{World Wide Web} o simplemente \textit{Web} es el sistema de información público más importante
desarrollado en los últimos 30 años, el cual permite la transmisión de documentos electrónicos identificados
por \texttt{URLs} \textit{(Uniform Resource Locators)}, los cuales pueden estar enlazados a otros documentos 
a través de \texttt{hipertexto} y que se encuentran disponibles utilizando servicios de la internet.

La \textit{World Wide Web} fue diseñada como un espacio para la información con el objetivo de no solo ser
útil para las comunicaciones entre humanos, sino que también un lugar donde las máquinas podrían ayudar y participar.
Sin embargo, uno de los principales problemas de la \textit{Web} es que la mayor parte de su contenido ha sido
diseñado para ser consumido por humanos, lo que implica que para las máquinas y el software no es fácil acceder 
e interpretar el contenido disponible, incluso si este proviene de una base de datos estructurada a través de
columnas claras y tipificadas. La web semántica busca desarrollar herramientas,
lenguajes, protocolos y estándares que permitan, tanto a maquinas como humanos, procesar toda la información disponible en la
\textit{Web}. En base a esto, podemos definir a la web semántinca como la idea de generar
una red de datos en la \textit{Web}, hasta cierto punto, una base de datos global. \cite{berners1998semantic}

    \subsubsection{Arquitectura de la web semántica}

[INSERTAR IMAGEN DE WIKIPEDIA]

    \subsection{RDF}

    RDF is...

    \subsection{SPARQL}

\bibliographystyle{IEEEtran}
\bibliography{IEEEabrv, propuesta.bib}

\end{document}

% refs: what is sparql
% refs: what is rdf
% refs: what is wikidata
% refs: ides, autocompletion features, intellisense